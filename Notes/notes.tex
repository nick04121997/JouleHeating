\documentclass[12pt]{article}

\usepackage[T1]{fontenc}
\usepackage{tikz}
\usepackage{float}
\usepackage{amsmath}
\usepackage{geometry}
\usepackage{siunitx}

\setlength\parindent{0pt}
\geometry{top=1in}

\begin{document}
\section{Attempt 1}
We have the following scenario.
\begin{figure}[H]
\centering
\begin{tikzpicture}
\node at (2,4.5) {$V(1) = 115 \si{\volt}$};
\draw (0,0) -- (4,0) -- (4,4) -- (0,4) -- (0,0);
\node at (2,-0.5) {$V(0) = -115 \si{\volt}$};
\node at (2,2) {$\delta(x,y)$};
\node [rotate=-90] at (4.5,2) {$\vec{J}=0$};
\node [rotate=90] at (-.5,2) {$\vec{J}=0$};

\end{tikzpicture}
\end{figure}
With the problem setup we then continue onto the equations to solve.
\begin{gather}
\nabla \boldsymbol{\cdot} (\sigma \delta(x,y) \nabla V) = 0 \\
q = \sigma|\nabla V|^2 \delta
\end{gather}
The overall system is defined as follows
\begin{figure}[H]
\centering
\begin{tikzpicture}
\draw (0,0) rectangle (2.5,2.5);
\node at (1.25,1.25) {Solver};
\node (A1) at (-2,0.5) {$q^*$};
\node (B1) at (-2,2.0) {$\delta_{initial}$};
\node (A2) at (-.25,0.5) {};
\node (B2) at (-.25,2.0) {};
\draw [thick,->] (A1) -- (A2);
\draw [thick,->] (B1) -- (B2);
\node (C1) at (2.5,1.25) {};
\node (C2) at (4,1.25) {$\delta_{output}$};
\draw [thick,->] (C1) -- (C2);
\end{tikzpicture}
\end{figure}
Suppose we had the following sample data for our $q^*$ desired profile and the guess initial thickness profile which is constant.
\begin{table}[H]
\centering
\begin{tabular}{c c c c}
\hline \hline
x & y & $q^* [\si{\watt \per \meter^2}]$ & $\delta_{1} [\si{\meter}]$ \\ [0.5ex] 
\hline
0 & 0 & 1000 & $4.1\cdot10^{-8}$ \\
.5 & 0 & 1000& $4.1\cdot10^{-8}$\\
1 & 0 & 1000& $4.1\cdot10^{-8}$\\
0 & 0.5 & 2000& $4.1\cdot10^{-8}$\\
.5 & 0.5 & 2000& $4.1\cdot10^{-8}$\\
1 &0.5 & 2000& $4.1\cdot10^{-8}$\\
0 &1 & 3000& $4.1\cdot10^{-8}$\\
.5 &1 & 3000& $4.1\cdot10^{-8}$\\
1 &1 & 3000& $4.1\cdot10^{-8}$\\
\end{tabular}
\end{table}
\newpage
Since this is a simple thickness profile $\left(\delta(x,y) = 4.1\cdot 10^{-8}\right)$ we can find an analytical expression.
\begin{align*}
&\nabla \boldsymbol{\cdot} (\sigma \delta(x,y) \nabla V) = \sigma \delta(x,y) \frac{\partial^2 V}{\partial^2 y} = 0\\
&\Longrightarrow \sigma \delta(x,y) \frac{\partial V}{\partial y} = c \;\;\; \textrm{(some constant c)} \\ 
&\Longrightarrow \partial V = \frac{c}{\sigma \delta(x,y)} \partial y\\
&\Longrightarrow \int_{y=0}^{y} \partial V = \int_{y=0}^{y} \frac{c}{\sigma \delta(x,y)} \partial y \\
&\Longrightarrow V(y) - V(0) = \left. \frac{c }{\sigma \delta(x,y)}y\right|_{y=0}^y \\
&\Longrightarrow V(y) = V(0) + \frac{c}{\sigma \delta(x,y)}y
\end{align*}
From our boundary conditions we see that $V(1) = 115$ and $V(0) = -115$.
Assume $\sigma = 10^6$.
\begin{gather*}
V(1) = 115 = -115 + \frac{c }{10^6\cdot(4.1\cdot10^{-8})}\cdot 1  \\
\Rightarrow c = 9.43
\end{gather*}
Therefore for the initial iteration we get that our voltage function is 
\[
V_1(y) = -115 + \frac{9.43\cdot y}{10^6 \cdot (4.1\cdot 10^{-8})}
\]
We can also calculate our surface joule heating function for this iteration
\begin{align*}
q_1 &= \sigma|\nabla V_1|^2 \delta_1 \\
&= \sigma \left(\frac{\partial V_1}{\partial y}\right)^2 \delta_1 \\
&= 10^6 \left(\frac{9.43}{10^6\cdot(4.1 \cdot 10^{-8})}\right)^2 (4.1 \cdot 10^{-8})\\
&= \frac{9.43^2}{10^6\cdot(4.1\cdot10^{-8})} = 2168.9 [w/m^2]
\end{align*}
\newpage
So if we perform our typical division to update the thicknesses for the next iteration as follows
\begin{align*}
\delta_{i+1} &= \frac{q^*}{Q_i} \\
&= \frac{q^*}{\frac{q_i}{\delta_i}} \\
&= \frac{q^* \cdot \delta_i}{q_i}
\end{align*}
We then get the following new table of values
\begin{table}[H]
\centering
\begin{tabular}{c c c c}
\hline \hline
x & y & $q^* [\si{\watt \per \meter^2}]$ & $\delta_{2} [\si{\meter}]$ \\ [0.5ex] 
\hline
0 & 0 & 1000 & $1.89\cdot10^{-8}$ \\
.5 & 0 & 1000& $1.89\cdot10^{-8}$\\
1 & 0 & 1000& $1.89\cdot10^{-8}$\\
0 & 0.5 & 2000& $3.78\cdot10^{-8}$\\
.5 & 0.5 & 2000& $3.78\cdot10^{-8}$\\
1 &0.5 & 2000& $3.78\cdot10^{-8}$\\
0 &1 & 3000& $5.67\cdot10^{-8}$\\
.5 &1 & 3000& $5.67\cdot10^{-8}$\\
1 &1 & 3000& $5.67\cdot10^{-8}$\\
\end{tabular}
\end{table}
We then perform a linear fit on the $\delta_2$ values to get a function that is linear in $y$ or that $\delta_2(x,y) \propto y$.
However when we solve for this joule heating we find that a linear fit in the thicknesses correspond to a inverse relationship for the surface joule heating.
\[
\delta_2(x,y) \propto y \rightarrow q_2 \propto \frac{1}{y}
\]
So when we use the values associated with this new $q_2$ to do our division update, it makes the region near $y=0$ have a smaller thickness.
Decreasing the thickness will actually raise the joule heating in that section and this is not what we want since $y(0)$ should have the lowest joule heating.
\\
We are concerned that the expression
\[
\delta_{i+1} = \frac{q^*}{Q_i}
\]
is not doing what we want.
\pagebreak



\section{Attempt 2}
We will try to linearize some of our quantities and use first order approximations.
\begin{align*}
\delta &= \overline{\delta} + \widetilde{\delta}\\
V &= \overline{V} + \widetilde{V} \\
q &= \overline{q} + \widetilde{q} 
\end{align*}
So we start with our two original equations
\begin{gather*}
\nabla \cdot (\sigma \delta \nabla V) = 0 \\
q = \sigma |\nabla V|^2 \delta
\end{gather*}
then we take derivatives and appropriately substitute our first order derivatives.
Doing this we get
\begin{gather*}
\widetilde{q} = 2(\nabla \overline{V} \cdot \nabla \widetilde{V}) \overline{\delta} + |\nabla \overline{V}|^2 \widetilde{\delta} \\
\nabla \cdot (\widetilde{\delta}\nabla \overline{V}+ \overline{\delta} \nabla \widetilde{V}) = 0
\end{gather*}
We use the first equation to solve for the error in thickness, $\widetilde{\delta}$.
\[
\widetilde{\delta} = \frac{\widetilde{q} - 2(\nabla \overline{V} \cdot \nabla \widetilde{V}) \overline{\delta}}{|\nabla \overline{V}|^2 }
\]
We substitute this into the second equation to get
\[
\nabla \cdot \left(\frac{\widetilde{q} - 2\left(\nabla \overline{V} \cdot \nabla \widetilde{V} \right) \overline{\delta}}{|\nabla \overline{V}|^2 }\nabla \overline{V}+ \overline{\delta} \nabla \widetilde{V}\right) = 0
\]
We then separate the term and simplify yielding
\begin{gather*}
-\nabla \cdot \left( \widetilde{q} \frac{\nabla \overline{V}}{|\nabla \overline{V}|^2}\right) 
=
\nabla \cdot \left(\overline{\delta}\nabla \widetilde{V} -2 \overline{\delta} \frac{\nabla \overline{V}\cdot \nabla \widetilde{V}}{|\nabla \overline{V}|^2}\nabla\overline{V} \right) \\
= \sum_{i=1}^2\sum_{j=1}^2 \nabla_i Q_{i,j} \nabla_j \widetilde{V} \\
= 
\frac{\partial}{\partial x}\left(Q_{1,1}\frac{\partial \widetilde{V}}{\partial x}\right) 
+
\frac{\partial}{\partial x}\left(Q_{1,2}\frac{\partial \widetilde{V}}{\partial y}\right)
+
\frac{\partial}{\partial y}\left(Q_{2,1}\frac{\partial \widetilde{V}}{\partial x}\right)
+
\frac{\partial}{\partial y}\left(Q_{2,2}\frac{\partial \widetilde{V}}{\partial y}\right)
\end{gather*}
where 
\[
Q = \overline{\delta}
\begin{bmatrix}
\displaystyle 1-\frac{2V_x^2}{|\nabla V|^2} & \displaystyle -\frac{2V_xV_y}{|\nabla V|^2} \\[1em]
\displaystyle -\frac{2V_xV_y}{|\nabla V|^2} & \displaystyle 1-\frac{2V_y^2}{|\nabla V|^2}
\end{bmatrix}
\]
So the idea is that we follow the following steps
\begin{enumerate}
\item 
Choose $\delta_n$
\item
Solve for $V_n$ using the equation $\nabla \cdot (\sigma \delta \nabla V) = 0$ and FEA in MATLAB
\item
Calculate the updated Joule heating
\[
q_n = \sigma |\nabla V_n|^2 \delta_n
\]
\item
Calculate the error in joule heating
\[
\widetilde{q} = q_n - \overline{q}
\]
where $\overline{q}$ is our desired Joule heating.
If this difference is within some tolerance, terminate.
Else continue on to the next step.
\item
We then solve the new PDE here
\[
-\nabla \cdot \left( \widetilde{q} \frac{\nabla V_n}{|\nabla V_n|^2}\right) 
=
\nabla \cdot \left(\delta_n \nabla \widetilde{V} -2 \overline{\delta} \frac{\nabla {V_n}\cdot \nabla \widetilde{V}}{|\nabla {V_n}|^2}\nabla {V_n} \right) 
\]
and this calculates an error in the voltage given by $\widetilde{V}$.
\item
Calculate the new change in delta using the expression
\[
\widetilde{\delta} = \frac{\widetilde{q}}{|\nabla {V_n}|^2}-\frac{2\nabla {V_n}\cdot \nabla \widetilde{V}}{|\nabla {V_n}|^2}\delta_n
\]
\item
We can then calculate the new thickness using 
\[
\delta_{n+1} = \delta_n + \widetilde{\delta}
\]
and use this as our new thickness then go back to step 2 and repeat through.
\end{enumerate}
\end{document}





